
\documentclass[12pt]{report}

\input{preamble}
\input{macros}
\setlength\parindent{0pt}

\title{\Huge{}\\}
\author{\huge{}}
\date{}

\begin{document}
% \maketitle
% \newpage% or \cleardoublepage
% \pdfbookmark[<level>]{<title>}{<dest>}
\pdfbookmark[section]{\contentsname}{toc}
% \tableofcontents
% \pagebreak
\chapter{}
\section{Phases}
Project lifecycle
\begin{enumerate}
  \item Initiation
        \begin{itemize}
                \item Projects objective or need is identified
                \item ``Can/should we do the project''
                \item Roles are appointed
        \end{itemize}
  \item Planning
        \begin{itemize}
          \item Heaviest step for the project manager
          \item Necessary steps are specified
          \item Teams identify work/task needed to be done
        \item Budget estimate costs
                \item Known as \textbf{Push approach}
                \item Identify the possible risks of the project and mitigations are put in place and communicated to stakeholders
                \item Using agile project management is called \textbf{Pull based approach}
        \end{itemize}
  \item Implementation
        \begin{itemize}
          \item Project is put into work
          \item Communication about progress is key both to project managers and stakeholders
                \item Regularized team meetings such as SCRUMS
        \end{itemize}
  \item Closure
        \begin{itemize}
          \item Releasing the final deliverables
          \item Handing documentation over to business
                \item Communication of finalization to stakeholders
        \end{itemize}
\end{enumerate}
Throughout the projects lifecycle we have ``mini-livecycles'' called \textbf{process groups} which are the following steps
\begin{enumerate}
  \item Initiating:
  \item Planning
  \item Executing
  \item Monitoring and controlling
        \begin{itemize}
          \item While work is being done on the project we need to monitor progress with
                \begin{itemize}
                  \item The plan
                  \item The requirements
                  \item The budget
                  \item The quality
                \end{itemize}
        \end{itemize}
  \item Closing

\end{enumerate}
\textbf{Project management star model}

\includegraphics[width=0.5\textwidth]{pm_star_model}

\subsection{PMBOK}
  Managing
  \begin{itemize}
    \item Integration
          \begin{itemize}
              \item Team defines scope of work
              \item Preliminary schedule
              \item Conceptual budget
              \item Team builds a plan for executing the project based on the project profile
              \item Plan for developing and tracking detailed schedule
              \item Plan for building budget and estimating and tracking costs
              \item IT plan (including communications)
              \item Diagrams, flowcharts and responsibilities matrices
          \end{itemize}
    \item Scope: What will be done defined in an execution plan
          \begin{itemize}
                  \item Description of the scope
                  \item Product acceptance criteria
                  \item Project deliverables
                  \item Project exclusions
                  \item Project constraints
                  \item Project assumptions
          \end{itemize}
          \nt{Change in the scope of a project can be very costly so we must manage and prepare for inevitable change}
    \item Time/Schedule
          \begin{itemize}
            \item Schedule and incremental milestones in accordance with key features/dates
            \item The longest path to completion is called the \textbf{Critical path}
                  \item If critical path is shorter than the alloted time: \textit{Positive float or slack}
            \item If critical path is longer than alloted time: \textit{Negative float}

          \end{itemize}
    \item Costs
          \begin{itemize}
                  \item Ballpark estimates increase in accuracy with more experience with previous projects and their associated actual versus estimated budgets
                  \item As the project proceeds we must compare the estimate against actual costs and if there is deviation corrective action must be taken and shared with stakeholders
          \end{itemize}
    \item Quality
    \item Human Resources
          \begin{itemize}
            \item Generally there are 2 types of members
                  \begin{itemize}
                    \item Functional managers and team who work on the tech and development of the project
                          \item Process managers and team focus on the ``Project'' side of things such as planning, costs, and communication
                  \end{itemize}
          \end{itemize}
    \item Communication
          \begin{itemize}
                  \item Regularized meetings to communicate progress and if it is going according to plan and if there is any corrective action needed
          \end{itemize}
    \item Risk:
          \begin{itemize}
              \item The likelyhood that an event will happen during the life of the project that will negatively affect the achivement of the project goals
                  \item Project risk plan reflects the risk profile of the project and balances the investment of the mitigation against the benefit of the project
                  \item Periodic risk-plan reviews during the life of the project
          \end{itemize}
    \item Procurement: External services/vendors
    \item Stakeholders
  \end{itemize}

\textbf{Waterfall Project lifecycle}

\includegraphics[width=0.5\textwidth]{waterfall_plcycle}

\section{Stakeholders}

\end{document}
